\header{chapter1}
\chapter{}
\label{chapter1}
\mkexercise{La densidad del cobre es $d = 8.96 \ [gr/cm^3]$ y su
peso atómico es de $p_a = 63.546 \ [gr/mol]$. Calcular la densidad
electrónica del cobre o el número de electrones móviles por centímetro
cúbico.}

\mksolution{
  \begin{equation*}
    \begin{split} \dfrac{\#e}{cm^3} &= \dfrac{d \cdot
        N_a}{p_a} = \dfrac{8.96 \ [gr/cm^3] \cdot 6.022\cdot10^{22} \
        [\#e/mol]}{63.54 \ [gr/mol]} \\ \dfrac{\#e}{cm^3} &= 8.49 \cdot
      10^{22} [\#e/cm^3]
    \end{split}
  \end{equation*}}

\mkexercise{Una corriente de
  electrones con una intensidad $I = 500 \ [mA]$ circula por un hilo de
  cobre cilíndrico largo de $2 \ [mm]$ de diámetro. Calcular la
  velocidad de deriva o de grupo $\vv{V_m}$ con la que viajan los
electrones. La carga del electrón es $q_e = 1.6 \cdot 10^{-19} \
[C]$.}

\mksolution{
  \begin{equation*}
    \begin{split} j &= \dfrac{I}{A}\\ j &= qNV_m\\ V_m &=
      \dfrac{I}{qNA}\\ V_m &= \dfrac{500 \cdot 10^-3}{(1,6 \cdot
        10^{-19})(8.49 \cdot 10^{22})(\pi(0.1)^2)}\\ V_m &= 0.001 \ [cm/s]
    \end{split}
  \end{equation*}}

En este ejercicio hay que hacer un pequeño cambio de unidades en el
área dado que la concentración de portadores
lo tenemos en $cm^3$.

\mkexercise{La conductividad eléctrica del cobre a temperatura ambiente es $\sigma = 5.96 \cdot 10^7 \
  [\Omega^{-1}m^{-1}]$. Calcular la movilidad del electrón en el cobre.}

\mksolution{
  \begin{equation*}
    \begin{split} \sigma &= qN\mu \\ \mu &= \dfrac{\sigma}{qN}
      \\ \mu &= \dfrac{5.96 \cdot 10^7 \ [\Omega^{-1} m^{-1}]}{(1,6 \cdot
        10^{-19}\ [A \cdot seg/e]))(8.49 \cdot 10^{28} \ [e/m^3])} \\ \mu &=
      0.0043 \ [m^2/Volt \cdot seg]
    \end{split}
  \end{equation*}}
\mkexercise{Un condensador formado por dos
  láminas metálicas paralelas de cobre con un dieléctrico de poliéster
  posee una capacidad de $1 \ [\mu F]$, y está conectado a una
  diferencia de potencial constante de $5 \ [V]$.
  \begin{itemize}
  \item Calcular el número de electrones por exceso (carga
    negativa) y por defecto (carga positiva) que se encuentran en la
    superficie de cada una de las láminas.
  \item Calcular la energía alamacenada en [Watios $\cdot$
    hora].
  \end{itemize}}

\mksolution{
  \begin{equation*}
    \begin{split} Q &= C (\Delta V)\\ Q &= (1 \ [\mu F]) (5 \
      [V])\\ Q &= 5 \cdot 10^{-6} \ [C]\\ Q &= 5 \cdot 10^{-6} \ [C]\\ e &=
      \dfrac{Q}{q_e}\\ e &= \dfrac{5 \cdot 10^{-6} \ [C]}{1.6 \cdot 10^{-19}
        \ [C/e]}\\ e &= 3.125 \cdot 10^{13}
    \end{split}
  \end{equation*}
        
  Observemos que es una cantidad relativamente pequeña comparada
  con la densidad electrónica del cobre que es de $8.49 \cdot 10^{22}
  [\#e/cm^3]$. Por esta razón, almacenar electrones en condensadores
  como método de almacenamiento de energía masiva no es buena idea.
  
  Aunque los pendrives no tienen ningún problema en almacenar
  bits mediante condensadores o eso se comenta en la clase...}

\mksolution{
  \begin{equation*}
    \begin{split} E_c '(Q) &= V\\ \int_0^Q E_c '(q)dq &= \int_0^Q
      v dq\\
    \end{split}
  \end{equation*} Usamos la ecuación $C = \dfrac{Q}{V_1-V_2}$  considerando que
  la tensión inicial o $V_2 = 0$.
  \begin{equation*}
    \begin{split} E_c(Q)-E_c(0) &= \int_0^Q \dfrac{q}{c} dq\\
      E_c(Q) &= \dfrac{1}{c} \int_0^Q q dq\\ E_c(Q) &= \dfrac{1}{2}
      \dfrac{Q^2}{c} \\ E_c(Q) &= \dfrac{(\micro{5})^2}{(\micro{1})(2)} =
      \nc{1.25}{-5} \ [Watts \cdot seg (Julios)]
    \end{split}
  \end{equation*}
  \begin{equation*} E_c = \dfrac{(\nc{1.25}{-5}[Watts \cdot
      seg])(1 \ [horas])}{3600 \ [seg]} = \nano{3.47} [Watts \cdot hora]
  \end{equation*}}
        
\mkexercise{Por una resistencia superficial de grafito
  (carbón) circula una corriente de $25 \ [mA]$, siendo la diferencia de
  potencial aplicada entre sus extermos de $5 \ [V]$.
  \begin{itemize}
  \item ¿Cuál es su resistencia en ohmios?
  \item Si la resistividad del grafito amorfo es de $1.6 \cdot
    10^{-5} \ [\Omega \cdot m]$ y la resistencia es de tipo superficial
    con un espesor $t = 1 \ [\mu m]$, y anchura de $W = 2 \ [mm]$. ¿Cuál
    es la longitud $L$ de la resistencia?
  \end{itemize}}

\mksolution{
  \begin{equation*}
    \begin{split} V&=RI\\ R &= \dfrac{5\ [V]}{\micro{25} \
        [A]} \\ R &= 200 \ [\Omega]
    \end{split}
  \end{equation*}}

\mksolution{
  \begin{equation*} R = \rho
    \dfrac{L}{A} = \rho \dfrac{L}{t\cdot W} = R_s\dfrac{L}{W}
  \end{equation*}
  \begin{equation*} R_s = \dfrac{\rho}{t} =
    \dfrac{\nc{1.6}{-5}}{\micro{1}} = 16 \ [Ohms]
  \end{equation*}
  \begin{equation*} L = \dfrac{R}{R_s} W = \dfrac{200 \
      [ohms]}{16 \ [ohms]} 2 \ [mm] = 25 \ [mm] = 2.5 \ [cm]
  \end{equation*}}
        
     
     