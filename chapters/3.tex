\header{chapter2}
\chapter{}
    \label{chapter2}
\mkexercise{Sea el circuito $ RC$ de la figura al que se le aplica un señal senoidal de 5 voltios de amplitud y una frecuencia de 1000 $Hz$: $V_i(t) =5\cos{(2\pi\cdot1000t+0)}$.Calcular la caída de tensión en el condensador $V_c$ y la corriente $I_c$ que circula por el condensador. Verificar los resultados mediante simulación.}
\mkexercise{Sea el circuito $ RL$ de la figura al que se le aplica un señal senoidal de 5 voltios de amplitud y una frecuencia de 1000 $Hz$: $V_i(t) =5\cos{(2\pi\cdot1000t+0)}$.Calcular la caída de tensión en el inductor $V_L$ y la corriente $I_L$ que circula por el inductor. Verificar los resultados mediante simulación.}
\mkexercise{Estudiar y caracterizar una impedancia $RL$ paralelo}
\mkexercise{Estudiar y caracterizar una impedancia $RC$ paralelo}
\mkexercise{Analíticamente y mediante simulación, calcular la respuesta en frecuencia del siguiente circuito denominado ``Filtro pasivo Pasa-Baja de primer orden''.}
\mkexercise{Analíticamente y mediante simulación, calcular la respuesta en frecuencia del siguiente circuito denominado ``Filtro pasivo Pasa-Alta de primer orden''.}
\mkexercise{Analíticamente y mediante simulación, calcular la respuesta en frecuencia del siguiente circuito denominado ``Filtro pasivo Pasa-Baja con Polo y cero''.}
\mkexercise{Analíticamente y mediante simulación, calcular la respuesta en frecuencia del siguiente circuito denominado ``Filtro pasivo Pasa-Banda (RC\_Serie-RC\_Paralelo)''.}
\mkexercise{Analíticamente y mediante simulación, calcular la respuesta en frecuencia del siguiente circuito denominado ``Filtro pasivo Pasa-Todo''. Explicar su funcionamiento inútil.}
\mkexercise{Analíticamente y mediante simulación, calcular la respuesta en frecuencia del siguiente circuito.}

