\header{chapter1}
\chapter{}
\label{chapter1}

\mkexercise{Demostrar que la inversa de la capacidad equivalente de la
asociación de condensadores en serie es igual a a la suma de las
capacidades recíprocas.}

\mksolution{
  Aplicamos la segunda ley de Kirchhoff:
  \begin{equation}
    \begin{split}
      V_{eq}(t)- V_1(t)- V_2(t) - V_3(t) - ... - V_n(t) = 0 \\
      V_{eq}(t) = V_1(t) + V_2(t) + V_3(t) + ... + V_n(t) \\
      \end{split}
    \end{equation}
    Derivamos la función $V_{eq}(t)$ .
    \begin{equation}
        V_{eq}'(t) = V_1'(t) + V_2'(t) + V_3'(t) + ... + V_n'(t) \\
      \end{equation}
    Reemplazamos $V_n$ por la ecuación de la caída de tensión en los
    bornes de un condensador:
    \begin{equation}
    \dfrac{I(t)}{C}  =  V'(t)
    \end{equation}

    \begin{equation}
      \dfrac{I(t)}{C_{eq}} = \dfrac{I(t)}{C_{1}} + \dfrac{I(t)}{C_{2}} +
      \dfrac{I(t)}{C_{3}} + ... + \dfrac{I(t)}{C_{n}}
    \end{equation}

    Como la corriente que pasa por el circuito es igual podemos
    eliminarla de la ecuación y resulta:
    \begin{equation}
      \dfrac{1}{C_{eq}} = \dfrac{1}{C_{1}} + \dfrac{1}{C_{2}} +
      \dfrac{1}{C_{3}} + ... + \dfrac{1}{C_{n}}
      \end{equation}
  }


\mkexercise{Calcula el equivalente Thevening entre los puntos A y B
  del circuito siguiente:}

\mksolution{
  \begin{enumerate}[1.]
    \item Eliminar fuentes de corriente y tensión. Las de corriente
      se transforman en circuitos abiertos y las de tensión en
      cortocircuitos.

      %% Insertar imagen resultante Esto simplifica mucho el circuito
      Ya que los cortocircuitos eliminan algunas resistencias, como $R_1$ y
      $R_2$. \footnotetext[1]{\textbf{ NOTA:} Hay que recordar que la asociación de
        resistencias no funciona igual porque el puerto A-B está desconectado
        pero tiene un carga asociada.}

     \begin{equation}
       \begin{split}
         R_s &= R_4+R_3+R_5\\
         R_S &= \nc{3}{3} + \nc{5}{3} + 500\\
         R_s &=  \nc{8.5}{3}\\
         R_{TH} &= \dfrac{1}{\dfrac{1}{R_s}+\dfrac{1}{R_6}}\\
         R_{TH} &= \dfrac{1}{\dfrac{1}{\nc{8.5}{3}}+\dfrac{1}{\nc{0.8}{3}}}\\
         R_{TH} &= 731.18 \  \Omega
         \end{split}
       \end{equation}
       \item Para calcular la $V_{TH}$ tenemos que resolver
         el circuito y calcular la diferencia de potencial
         entre los terminales A y B.
   \end{enumerate}
}

\mkexercise{Determinar el punto de operación (bias point), del
siguiente circuito de forma manual y después verificarlo con la ayuda
del simulador.}

\mkexercise{Con la ayuda del simulador calcular la curva de
transferencia:''Tensión de salida en función de la tensión de
entrada" para el circuito de la figura considerando que su entrada
cambia entre 0 y 5 votios. ¿Qué función hace el
condensador?. ¿Qué tensión hay a la salida en ausencia de tensión
de entrada?}

\mkexercise{Con la ayuda del simulador, estudiar y explicar el
comportamiento Transitorio del Circuito de la figura durante 40
milisegundos cuando se excita con un pulso de 5 voltios durante un
tiempo de 10 milisegundos. ¿Cuál es la carga almacenada en el
condensador a los 5 milisegundos?.¿Y a los 30 milisegundos?.}