\header{chapter1}
\chapter{}
    \label{Relación de problemas 1}
\mkexercise{Demostrar que la inversa de la capacidad equivalente de la asociación de condensadores en serie es igual a a la suma de las capacidades recíprocas.} 
\mkexercise{Calcula el equivalente Thevening entre los puntos A y B del circuito siguiente:}
\mkexercise{Determinar el punto de operación (bias point), del siguiente circuito de forma manual y después verificarlo con la ayuda del simulador.}
\mkexercise{Con la ayuda del simulador calcular la curva de transferencia:''Tensión de salida en función de la tensión de entrada" para el circuito de la figura considerando que su entrada cambia entre 0 y 5 votios. ¿Qué función hace el condensador?. ¿Qué tensión hay a la salida en ausencia de tensión de entrada?}
\mkexercise{Con la ayuda del simulador, estudiar y explicar el comportamiento Transitorio del Circuito de la figura durante 40 milisegundos cuando se excita con un pulso de 5 voltios durante un tiempo de 10 milisegundos. ¿Cuál es la carga almacenada en el condensador a los 5 milisegundos?.¿Y a los 30 milisegundos?.}