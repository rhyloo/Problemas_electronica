\header{chapter2}
\chapter{}
\label{chapter2}

\mkexercise{Demostrar que la inversa de la capacidad equivalente de la
asociación de condensadores en serie es igual a a la suma de las
capacidades recíprocas.}

\mksolution{
  Aplicamos la segunda ley de Kirchhoff:
  \begin{equation}
    \begin{split}
      V_{eq}(t)- V_1(t)- V_2(t) - V_3(t) - ... - V_n(t) = 0 \\
      V_{eq}(t) = V_1(t) + V_2(t) + V_3(t) + ... + V_n(t) \\
      \end{split}
    \end{equation}
    Derivamos la función $V_{eq}(t)$ .
    \begin{equation}
        V_{eq}'(t) = V_1'(t) + V_2'(t) + V_3'(t) + ... + V_n'(t) \\
      \end{equation}
    Reemplazamos $V_n$ por la ecuación de la caída de tensión en los
    bornes de un condensador:
    \begin{equation}
    \dfrac{I(t)}{C}  =  V'(t)
    \end{equation}

    \begin{equation}
      \dfrac{I(t)}{C_{eq}} = \dfrac{I(t)}{C_{1}} + \dfrac{I(t)}{C_{2}} +
      \dfrac{I(t)}{C_{3}} + ... + \dfrac{I(t)}{C_{n}}
    \end{equation}

    Como la corriente que pasa por el circuito es igual podemos
    eliminarla de la ecuación y resulta:
    \begin{equation}
      \dfrac{1}{C_{eq}} = \dfrac{1}{C_{1}} + \dfrac{1}{C_{2}} +
      \dfrac{1}{C_{3}} + ... + \dfrac{1}{C_{n}}
      \end{equation}
  }


\mkexercise{Calcula el equivalente Thevening entre los puntos A y B
  del circuito siguiente:}

\mksolution{
  \begin{enumerate}[1.]
    \item Eliminar fuentes de corriente y tensión. Las de corriente
      se transforman en circuitos abiertos y las de tensión en
      cortocircuitos.

      %% Insertar imagen resultante Esto simplifica mucho el circuito
      Ya que los cortocircuitos eliminan algunas resistencias, como $R_1$ y
      $R_2$. \footnotetext[1]{\textbf{ NOTA:} Hay que recordar que la asociación de
        resistencias no funciona igual porque el puerto A-B está desconectado
        pero tiene un carga asociada.}

     \begin{equation}
       \begin{split}
         R_s &= R_4+R_3+R_5\\
         R_S &= \nc{3}{3} + \nc{5}{3} + 500\\
         R_s &=  \nc{8.5}{3} \ \Omega \\
         R_{TH} &= \dfrac{1}{\dfrac{1}{R_s}+\dfrac{1}{R_6}}\\
         R_{TH} &= \dfrac{1}{\dfrac{1}{\nc{8.5}{3}}+\dfrac{1}{\nc{0.8}{3}}}\\
         R_{TH} &= 731.18 \  \Omega
         \end{split}
       \end{equation}
       \item Para calcular la $V_{TH}$ tenemos que resolver
         el circuito y calcular la diferencia de potencial
         entre los terminales A y B.

         Planteamos el sistema de ecuaciones:
         \begin{equation}
           \begin{split}
             0 &= V_{fc}-i_{fc}R_1-V_1-i_1R_3-i_{fc}R_2 \\
             0 &= V_{fc}-i_{fc}R_1-i_2R_4-i_2R_6-V_2-i_2R_5-i_{fc}R_2 \\
             0 &= i_{fc}-i_2-i_1 \\
           \end{split}
         \end{equation}
       \end{enumerate}

       Las ecuaciones anteriores pueden reescribirse de la forma $Ax=B$.

       \begin{equation}
         \begin{bmatrix}
           1 & -5000 & 0 \\
           1 & 0 & -4300 \\
           0 & -1 & -1 
         \end{bmatrix}
         \begin{bmatrix}
           V_{fc}\\
           i_1\\
           i_2
         \end{bmatrix}
         =
         \begin{bmatrix}
           12\\
           15\\
           -\mili{3}
         \end{bmatrix}
       \end{equation}

       Podemos usar cualquier método para resolver el sistema, a
       efectos prácticos \textsc{MATLAB} es la
       solución ideal ya que con el comando
       \textit{mldivide(A,B)} lo hacemos
       fácilmente.

       Las soluciones para $V_{fc},i_1,i_2$
       son las siguientes:
       \begin{equation}
         \begin{split}
           V_{fc}&=20.5458 \\
           i_1&=\mili{1.7}\\
           i_2&=\mili{1.3}
           \end{split}
       \end{equation}

       No debemos olvidar que estas soluciones
       realmente no nos sirven directamente pero
       son el camino para calcular $V_{TH}$.
       Planteamos un último sistema de ecuaciones,
       que puede solucionarse por sustitución.

       \begin{equation}
         \begin{split}
           V_B&=6+V_c\\
           V_C &= (\mili{1.3})(500)\\
           V_A &= (\nc{0.8}{3})(\mili{1.3})+6+(\mili{1.3})(500)\\
           V_A &= 7.69 \ V
         \end{split}
       \end{equation}

       %%Necesito insertar imágenes aquí porque
       %%no se entiende dónde aplico V_B o V_C

       Para terminar de calcular la $V_{TH}$
       tenemos que calcular la diferencia de
       potencial entre $V_A$ y $V_C$.

       \begin{equation}
         \begin{split}
           V_{TH}&=V_A-V_C\\
           V_{TH}&=7,69 - 0.65\\
           V_{TH} &= 7,04 \ V
           \end{split}
       \end{equation} 

      \item Hemos obtenido la $V_{TH}$ y la
        $R_{TH}$.

        \begin{equation}
          \begin{split}
            V_{TH} = 7.04 \ V\\
            R_{TH} = 731.18 \ \Omega\\
            I_{cc} = \mili{9.63} \ A
            \end{split}
          \end{equation}
}

\mkexercise{Determinar el punto de operación (bias point), del
siguiente circuito de forma manual y después verificarlo con la ayuda
del simulador.}

\mksolution{
  Este circuito se puede resolver mediante la
  segunda ley de Kirchhoff, las ecuaciones son:

  \begin{equation}
    \begin{split}
      0 &= V_1-i_2R_1-V_3-i_2R_6-V_2\\
      0 &=V_1-i_3(R_3+R_4+R_5)-V_2\\
      0 &= i_1-i_2-i_3
      \end{split}
    \end{equation}

    Como podemos observar en las ecuaciones
    anteriores son independientes y pueden
    resolverse sin hacer uso de sistemas, no estoy
    seguro del motivo pero deduzco que es por la
    concepción topológica que tiene el circuito ya
    que simplifcado es un divisor de corriente.

    \begin{equation}
      \begin{split}
        i_1 &=i_2+i_3\\
        i_2 &= \dfrac{V_1-V_2}{R_3+R_4+R_5}\\
        i_3 &= \dfrac{V_1-V_2-V_3}{R_1+R_6}
      \end{split}
    \end{equation}

    Realmente lo anterior no es el bias point del
    circuito pero calcular las tensiones en cada
    nudo/nodo es relativamente sencillo con toda
    la información (corrientes encontradas).

    \begin{equation}
      \begin{split}
        i_1 &= 2.833 \ mA \\
        i_2 &= 5.833 \ mA \\
        i_3 &= 3.000 \ mA \\
      \end{split}
    \end{equation}

}

\mkexercise{Con la ayuda del simulador calcular la curva de
transferencia:``Tensión de salida en función de la tensión de
entrada'' para el circuito de la figura considerando que su entrada
cambia entre 0 y 5 votios. ¿Qué función hace el
condensador?. ¿Qué tensión hay a la salida en ausencia de tensión
de entrada?}

%\mksolution{
%Este ejercicio lo he resuelto con un par de
%cuentas más, he considerado efectos transitorios.

%Antes de empezar con la parte
%analítica/simulación, analicemos lo que debería
%ocurrir.

%Hay un condensador en el circuito eso implica ya
%sea en DC o AC una carga o descarga, una constante
%de tiempo y dos tramos de salida, la zona
%transitoria y la estacionaria.
%Si $V_{in}$ es cero la única carga que puede tener
%el condensador es la proporcionada por la pila de
%5 voltios

%}


\mkexercise{Con la ayuda del simulador, estudiar y explicar el
comportamiento Transitorio del Circuito de la figura durante 40
milisegundos cuando se excita con un pulso de 5 voltios durante un
tiempo de 10 milisegundos. ¿Cuál es la carga almacenada en el
condensador a los 5 milisegundos?. ¿Y a los 30 milisegundos?.}