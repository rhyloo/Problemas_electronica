\chapter{chapter1}

\begin{enumerate}[I)]
\item Determinar analíticamente las corrientes y las caídas de
  potencial en el nodo central y en la fuente de corriente del
  circuito de la figura, siendo el valor de la resistencia $R_4$ el
  indicado en ohmios por las tres cifras menos significativas del
  documento nacional de identidad o documento identificativo equivalente.

    \item Verificar los resultados analíticos con el simulador mediante un análisis ``bias point''.
    \item Realizar un informe explicativo con los apartados a) y b) y subirlos al campus a la tarea destinada para tal fin.
\end{enumerate}

\begin{figure}[H]
    \centering
\tikzset{poles/.style={
circuitikz/voltage/american plus=\textcolor{red}{$+$},
circuitikz/voltage/american minus=\textcolor{red}{$-$},
}}

\begin{circuitikz}[american]
\draw (0,0) -- (0,8) (0,8) -- (8,8) (8,8)--(8,0) (8,0)--(0,0);
\draw (4,8) to[isource, l=$10mA$, a_=$I_0$] (4,6) to[R, l2^=$R_1$ and \SI{270}{\ohm}] (4,4) to[R, l2_=$R_4$ and \SI{948}{\ohm}] (6,4);
\draw (8,4) to[battery,v_=$12V$] (6,4);
\draw (0,4) to[battery,v=$3V$] (2,4) to[R, n=res, l_=$R_2$] (4,4);
\draw (4,4) to[R, l2_=$R_3$ and \SI{330}{\ohm}] (4,0);
\draw (4,0) node[ground](GND){};
\draw (res.n) node[above]{150$\Omega$};
\end{circuitikz}
\caption{Circuito objetivo}
\label{circuito_problema}
\end{figure}

Como podemos observar en la figura \ref{circuito_problema} tenemos 4 \textbf{nodos}, 4 \textbf{ramas} y $n$ \textbf{mallas}, es decir que tenemos que generar cuatro ecuaciones\footnote{Para determinar las corrientes y caídas de potencial.} según la ecuación $R-(n-1)$ y seleccionar 3 de esas $n$ mallas porque la tercera ecuación la obtenemos del primer lema de Kirchhoff. 

En la figura \ref{circuito_solucion1} y \ref{circuito_solucion2} podemos observar los nodos, las polaridades\footnote{Recordemos que es una polaridad establecida para el análisis.} de los dispositivos,la dirección de las corrientes elegidas y las mallas seleccionadas.
\begin{figure}[H]
    \centering
\tikzset{poles/.style={
circuitikz/voltage/american plus=\textcolor{red}{$+$},
circuitikz/voltage/american minus=\textcolor{red}{$-$},
}}

\begin{circuitikz}[american, /tikz/circuitikz/bipoles/length=1cm]
\draw (0,0) -- (0,8) (0,8) -- (8,8) (8,8)--(8,0) (8,0)--(0,0);
\draw (4,8) to[isource,i^>=$i$,v_=$V_3$,*-,poles] (4,6) to[R,v_=$V_{R_1}$, -*,poles] (4,4) to[R,v^>=$V_{R_4}$,i_<=$i_1$,poles] (6,4);
\draw (8,4) to[battery, v^=$V_1$,*-, poles] (6,4);
\ctikzset{bipoles/generic/voltage/distance from node=0.8}
\draw (0,4) to[battery, v_=$V_2$,*-,poles] (2,4);
\draw (2,4) to[R,v_=$V_{R_2}$,i>^=$i_2$,poles] (4,4);
\ctikzset {voltage/distance from node=0.8}
\draw (4,4) to[R,i_>=$i_3$,v^=$V_{R_3}$, -*,poles] (4,0);
\draw (4,4) node[below right] {$A$};
\draw (0,4) node[left] {$B$};
\draw (4,0) node[above right] {$B$};
\draw (4,8) node[below right] {$B$};
\draw (8,4) node[right] {$B$};
\end{circuitikz}
\caption{Nodos y ``polaridad''}
\label{circuito_solucion1}
\end{figure}

\begin{figure}[H]
    \centering
\begin{circuitikz}[american]
\draw (0,0) -- (0,8) (0,8) -- (8,8) (8,8)--(8,0) (8,0)--(0,0);
\draw[blue,dashed,line width=2.5pt] (0,0) -- (0,8) -- (4,8) -- (4,0) -- (0,0);
\draw[red,line width=1.5pt] (0,4) -- (0,8) -- (4,8) -- (4,4) -- (0,4);
\draw[green,line width=1.5pt] (4,4) -- (4,8) -- (8,8) -- (8,4) -- (4,4);
\draw (4,8) -- (4,6) -- (4,4) -- (6,4);
\draw (8,4) -- (6,4);
\draw (4,4) -- (4,0);
\draw[->,shift={(6,6)},green] (120:.7cm) arc (120:-90:.7cm) node at(0,0){$M_2$};
\draw[->,shift={(2,2)},blue] (120:.7cm) arc (120:-90:.7cm) node at(0,0){$M_3$};
\draw[->,shift={(2,6)},red] (120:.7cm) arc (120:-90:.7cm) node at(0,0){$M_1$};
\end{circuitikz}
\caption{Mallas seleccionadas}
\label{circuito_solucion2}
\end{figure}

Las ecuaciones generadas son las siguientes\footnote{Para el segundo lema de Kirchhoff se considera $-+$ positivo y $+-$ negativo}:

\begin{minipage}[c]{10em}{
\begin{enumerate}[]
    \item $i=10 mA$
    \item $R_1=270 \Omega$
    \item $R_2=150 \Omega$
    \item $R_3=330 \Omega$
    \item $R_4=948 \Omega$
    \item $V_1=12 V$
    \item $V_2=3 V$
\end{enumerate}}
\end{minipage}
\hspace{5em}
\begin{minipage}[c]{20em}
\begin{equation}
i_3=i+i_2+i_1
\end{equation}
\begin{equation}
V_3-R_{1}(i)+R_{2}(i_{2})-V_2 = 0
\end{equation}
\begin{equation}
V_3-R_{1}(i)+R_{4}(i_{1})-V_1 = 0
\end{equation}
\begin{equation}
V_3-R_{1}(i)-R_{3}(i_{3}) = 0
\end{equation}
\end{minipage}

Reemplazamos en las ecuaciones los valores conocidos y obtenemos:
\begin{equation}
i_1+i_2-i_3=-\mili{10}
\end{equation}
\begin{equation}
V_3-2.7+(150)(i_{2})-3= 0
\end{equation}
\begin{equation}
V_3-2.7+(948)(i_{1})-12 = 0
\end{equation}
\begin{equation}
V_3-2.7-(330)(i_{3}) = 0
\end{equation}
Generamos un sistema de ecuaciones de la forma $Ax=B$.
\begin{equation}
\begin{bmatrix}
0 & 1 & 1 & -1\\
1 & 0 & 150 & 0\\
1 & 948 & 0 & 0\\
1 & 0 & 0 & -330
\end{bmatrix}
\begin{bmatrix}
V_3\\
i_1\\
i_2\\
i_3\\
\end{bmatrix}
=
\begin{bmatrix}
-\mili{10}\\
5.7\\
14.7\\
2.7\\
\end{bmatrix}
\end{equation}
Usamos el \textbf{método de Cramer} para resolver el sistema.
\begin{equation}
    \det(A) = 504540
\end{equation}
\begin{equation}
\left| \begin{bmatrix}
-\mili{10} & 1 & 1 & -1\\
5.7 & 0 & 150 & 0\\
14.7 & 948 & 0 & 0\\
2.7 & 0 & 0 & -330
\end{bmatrix}\right|=\det(A_{V_3}) = \nc{3.364}{6}
\end{equation}
\begin{equation}
\left| \begin{bmatrix}
0 & -\mili{10} & 1 & -1\\
1 & 5.7 & 150 & 0\\
1 & 14.7 & 0 & 0\\
1 & 2.7 & 0 & -330
\end{bmatrix}\right|=\det(A_{i_1}) = 4275
\end{equation}
\begin{equation}
\left| \begin{bmatrix}
0 & 1 & -\mili{10} & -1\\
1 & 0 & 5.7 & 0\\
1 & 948 & 14.7 & 0\\
1 & 0 & 2.7 & -330
\end{bmatrix}\right|=\det(A_{i_2}) = \nc{-3.25}{3}
\end{equation}
\begin{equation}
\left| \begin{bmatrix}
0 & 1 & 1 & -\mili{10}\\
1 & 0 & 150 & 5.7\\
1 & 948 & 0 & 14.7\\
1 & 0 & 0 & 2.7
\end{bmatrix}\right|=\det(A_{i_3}) = 6066
\end{equation}
\begin{equation}
    V_3 = \dfrac{\det(A_{V_3})}{\det(A)} = 6.667 \ V
\end{equation}
\begin{equation}
    i_1 = \dfrac{\det(A_{i_1})}{\det(A)} = 8.47 \ mA
\end{equation}
\begin{equation}
    i_2 = \dfrac{\det(A_{i_2})}{\det(A)} = -6.44 \ mA
\end{equation}
\begin{equation}
    i_3 = \dfrac{\det(A_{i_3})}{\det(A)} = 12 \ mA
\end{equation}
Hemos determinado el estado de todas las corrientes y tensiones. Solo he cometido un fallo en el supuesto de la corriente $i_2$ que va en dirección contraria.

Ahora podemos calcular la tensión del nodo central.
\begin{equation}
    V_{A} = (i_3)(R_3) = 3.9666 \ V
\end{equation}
Realizamos la simulación del circuito.
\vspace{-1em}
\begin{figure}[H]
\centering
%\includegraphics[clip, trim=3cm 4cm 10.5cm 13cm, scale=1,angle=270]{chapters/chapter_problemas/images/Problema puntuable 1.pdf}
\end{figure}
Como podemos observar en la figura los resultados son casi idénticos, las variaciones son apreciables en las milésimas.



